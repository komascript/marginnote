% \CheckSum{783}
% \iffalse meta-comment
% ======================================================================
% marginnote.dtx
% Copyright (c) Markus Kohm, 2005-2018
%
% This file is part of the work marginnote.
%
% This work may be distributed and/or modified under the conditions of
% the LaTeX Project Public License, version 1.3c of the license.
% The latest version of this license is in
%   http://www.latex-project.org/lppl.txt
% and version 1.3c or later is part of all distributions of LaTeX
% version 2005/12/01 or later and of this work.
%
% This work has the LPPL maintenance status "maintained".
%
% The Current Maintainer and author of this work is Markus Kohm.
%
% This work consists of the files marginnote.dtx and the
% derived files README.txt and marginnote.sty.
% ======================================================================
%
%<package>%%% From File: $Id$
%<*dtx>
\ifx\ProvidesFile\undefined\def\ProvidesFile#1[#2]{}\fi
\begingroup
  \def\filedate$#1: #2-#3-#4 #5${\gdef\filedate{#2/#3/#4}}
  \filedate$Date$
  \def\filerevision$#1: #2 ${\gdef\filerevision{#2}}
  \filerevision$Revision$
\endgroup
\ProvidesFile{marginnote.dtx}[\filedate\space\filerevision\space
%</dtx>
%<package>\NeedsTeXFormat{LaTeX2e}[1995/12/01]
%<package>\ProvidesPackage{marginnote}[%
%<README>LaTeX package marginnote
%<README>Copyright (c) Markus Kohm, 2005-2018
%<README>-------------------------------------------------------------------
%<README>Version:
% \fi^^A meta-comment
% \newcommand*{\packagedateandversion}{%
% \iffalse meta-comment
%<*package|README>
% \fi^^A meta-comment
  2018/07/21 v1.4a
% \iffalse meta-comment
%</package|README>
% \fi^^A meta-comment
% }
% \iffalse meta-comment
%<README>Licence:
%<README>  This work may be distributed and/or modified under the conditions 
%<README>  of the LaTeX Project Public License, version 1.3c of the license.
%<README>  The latest version of this license is in
%<README>    http://www.latex-project.org/lppl.txt
%<README>  and version 1.3c or later is part of all distributions of LaTeX
%<README>  version 2005/12/01 or later and of this work.
%<README>Abstract:
%<README>  This package provides the command \marginnote that may be used
%<README>  instead of \marginpar at almost every place, where \marginpar
%<README>  cannot be used, e.g., inside floats, footnotes, frames made with
%<README>  framed package.  See marginnote.pdf for more information.
%<README>-------------------------------------------------------------------
%<*dtx|package>
  non floating margin notes for LaTeX]
%</dtx|package>
%<*dtx>
\ifx\documentclass\undefined
  \input docstrip.tex

  \keepsilent
  \askforoverwritefalse
  \usedir{tex/latex/marginnote}

  \generate{%
    \file{marginnote.sty}{\from{marginnote.dtx}{package}}%
    \nopreamble\nopostamble
    \file{README.txt}{\from{marginnote.dtx}{README}}%
  }

  \ifToplevel{%
    \Msg{*********************************************************************}
    \Msg{*}
    \Msg{* marginnote}
    \Msg{* ==========}
    \Msg{*}
    \Msg{* To finish the installation run}
    \Msg{* \space\space pdflatex marginnote.dtx}
    \Msg{* After this copy}
    \Msg{* \space\space marginnote.sty to .../tex/latex/marginnote/}
    \Msg{* \space\space marginnote.pdf to .../doc/latex/marginnote/}
    \Msg{* \space\space marginnote.dtx to .../source/latex/marginnote/}
    \Msg{* \space\space README
      \space\space\space\space\space\space\space\space to 
      .../source/latex/marginnote/}
    \Msg{* where .../ is your local TDS tree}
    \Msg{*}
    \Msg{*********************************************************************}
  }
\else
  \let\endbatchfile\relax
\fi
\endbatchfile
\documentclass{ltxdoc}
\usepackage{graphicx}% needed for \reflectbox
\providecommand*{\XeTeX}{%
  X\kern-.1em\lower.5ex\hbox{\reflectbox{E}}\kern-.15em\TeX}
\CodelineIndex
\RecordChanges
\begin{document}
\DocInput{marginnote.dtx}
\end{document}
%</dtx>
%\fi
%
% \GetFileInfo{marginnote.dtx}
%
% \title{Non-Floating Margin Notes with \textsf{marginnote}
%   Package\thanks{This file has revision number \fileversion, last revised
%     \filedate.}}
% \author{Markus Kohm\thanks{Email: \texttt{komascript@gmx.info}}}
% \date{\packagedateandversion}
% \maketitle
%
% \begin{abstract}
%   In \LaTeX{} the command \cs{marginpar}\oarg{left}\marg{right} might be
%   used to create a note in the margin. But there is a problem with this
%   command: it creates a special kind of float. For this it cannot be used
%   e.g., at floats or footnotes. Package \textsl{marginnote} supports another
%   command \cs{marginnote} to create notes in the margin. This does not use a
%   kind of float and for this does not have the disadvantage of
%   \cs{marginpar}. But there might be other problems \dots
% \end{abstract}
%
% \tableofcontents
%
% \changes{v1.0b}{2006/14/03}{spelling fixes}
% \changes{v1.2b}{2017/10/14}{spelling fixes (by Thomas Reuben)}
%
% \section{How to Use \textsf{marginnote} Package}
%
% First of all you have to load. You may use:
% \begin{verbatim}
% \usepackage{marginnote}
% \end{verbatim}\vskip-\baselineskip
% to do so. You may also use one of the following options for a global change
% of the behaviour of \textsf{marginnote}:
% \begin{description}
% \item[\texttt{fulladjust}] adjusts the margin note at the height and depth
%   of the current line. Note, that this may sometimes result in extra height
%   and depth of the current line, but results in the best vertical
%   alignment. This is the default.
% \item[\texttt{heightadjust}] adjusts the margin note at the height of the
%   current line but not the depth. Note, that this may sometimes result in
%   extra height of the current line and in vertical misplacement.
% \item[\texttt{depthadjust}] adjusts the margin note at the depth of the
%   current line but not height. Note, that this may sometimes result in extra
%   depth of the current line and very often in vertical misplacement.
% \item[\texttt{noadjust}] does not adjust the margin note at the height or
%   depth of the current line. Note, that this often results in vertical
%   misplacement but seldom in vertical extra space before or after the
%   current line.
% \item[\texttt{parboxrestore}] uses \cs{@parboxrestore} to restore the
%   definition of \cs{par} and \cmd\\ and several other commands and sets
%   \cs{parindent} and \cs{parskip} to 0, \cs{parfillskip} to 0\,pt plus
%   1\,fil and \cs{lineskip} to \cs{normallineskip} and \cs{baselineskip} to
%   \cs{normalbaselineskip} for every margin note. This is the default since
%   release 1.4 of \textsf{marginnote}.
% \item[\texttt{noparboxrestore}] does not use \cs{@parboxrestore}.
% \end{description}
%
% \DescribeMacro{\marginnote}
% The command \cs{marginnote}\oarg{left}\marg{right}\oarg{voffset} may be used
% to set a margin note using \textsf{marginnote}. The first optional argument
% and the mandatory argument are same using \cs{marginpar} from the \LaTeX{}
% kernel. Even \cs{reversemarginpar} will be considered. The note \meta{left}
% or \meta{right} will be put at the current vertical position. Second
% optional argument \meta{voffset} may be used to adjust the vertical position
% of the margin note. Use a negative dimension to move it up or a positive
% dimension to move it down.
%
% \DescribeMacro{\marginnotetextwidth}
% Package \textsl{marginnote} needs to know the real width of the type area to
% find the right margin. While some environments (e.g., those of package
% \textsl{framed}) change \cs{textwidth}, \textsl{marginnote} defines its own
% text width macro. If you change type area after \cs{begin\{document\}} you
% should add
% \begin{verbatim}
%   \edef\marginnotetextwidth{\the\textwidth}
% \end{verbatim}\vskip-\baselineskip
% after changing the type area. Maybe you should do this globally using
% \verb|\xdef| instead of \verb|\edef|. Most users will never need to change
% \cs{marginnotetextwidth}.
%
% \DescribeMacro{\marginnotevadjust}
% At some environments the vertical adjustment of the margin note will be
% wrong, e.g., one base line to low. In this case you may use the additional
% optional argument of \cs{marginnote} at every usage of \cs{marginnote} or
% redefine \cs{marginnotevadjust} at the begin of the environment. The default
% definition is \texttt{0pt}.
%
% \DescribeMacro{\raggedleftmarginnote}
% \DescribeMacro{\raggedrightmarginnote}
% These macros define how the margin note will be aligned. The defaults are:
% \begin{itemize}
% \item align margin notes at the left margin right to the margin,
% \item align margin notes at the right margin left to the margin.
% \end{itemize}
% You may change this using \cs{renewcommand}, e.g., use^^A
% \changes{v1.0a}{2006/02/06}{Example to macros \cs{raggedleftmarginnote} and
%   \cs{raggedrightmarginnote} at documentation fixed [thanks to Susumu
%   Tanimura].}
% \begin{verbatim}
% \renewcommand*{\raggedleftmarginnote}{}
% \renewcommand*{\raggedrightmarginnote}{\centering}
% \end{verbatim}\vskip-\baselineskip
% to get justified text at the left and centered text at the right margin.
%
% \DescribeMacro{\marginfont}
% This macro defines the font that will be used to set margin notes. The
% default is \cs{normalcolor}. You may use \cs{renewcommand} to change this,
% e.g. use
% \begin{verbatim}
% \renewcommand*{\marginfont}{\color{red}\sffamily}
% \end{verbatim}\vskip-\baselineskip
% to get red colored margin notes in sans serif font family. You need to load
% e.g. package \textsf{color} to use \cs{color}.
%
%
% \section{Known Issues Using \textsf{marginnote}}
%
% From version 1.4a there is a workaround for double-ended documents with
% consecutive odd pages or consecutive even pages. However it is not
% recommended to use double-ended documents with such page sequences, because
% printing such documents could be a mess. \texttt{marginnote} shows a warning
% message whenever it detects those page sequences.
%
% From version 1.3 \texttt{marginnote} does not longer support \TeX{} engines
% without primitives \cs{pdfsavepos}/\cs{savepos} and
% \cs{pdflastxpos}/\cs{lastxpos}. The former (manual adjustment) fallback has
% been removed. You'll get an error message, if you try to use a \TeX{} engine
% without these primitives. Also $\varepsilon$-\TeX{} primitves are
% needed. However, with current free \TeX{} distributions like MiK\TeX{} or
% \TeX Live this shouldn't be a problem.
%
% At double side layout (e.g. using class option \texttt{twoside})
% \cs{marginnote} needs to know the number of the current page to decide
% whether the page is odd or even and so whether to use left or right
% margin. \LaTeX{} uses an asynchronous output. Because of this counter
% \texttt{page} should not be used to get the number of the current page
% unless you are at page head or foot. To solve the problem
% \textsf{marginnote} uses a mechanism similar to labels. But this means, that
% the correct margin won't be known at this \LaTeX{} run but only at the
% next. So after adding or deleting a margin note or after each change of page
% break you need two \LaTeX{} runs to get all margins right.
%
% The command \cs{marginnote} uses \cs{strut} and \cs{vadjust} to put the
% margin note at the correct position. But under some circumstances this may
% fail. You may adjust the vertical position of the margin note using the
% second optional argument of \cs{marginnote}. Sometimes even the text outside
% \cs{marginnote} will be moved because of using \cs{marginnote}. You may use
% one of the package options \texttt{fulladjust}, \texttt{heightadjust},
% \texttt{depthajust} or \texttt{noajust} to change the global adjustment or
% a local redefinition of |\mn@strut| or |\mn@zbox|.
%
% Note: The margin note will be placed at the current vertical line. This
% means, if you are using two \cs{marginnote} commands at the same line, they
% will be put on the same place. This is not a bug but a feature!
%
% Since release~1.1b \cs{marginnote} between paragraphs (in vertical mode)
% will place the note between the paragaphs instead of the end of the previous
% paragraph. You may use \cs{leavevmode} or the third optional argument of
% \cs{marginnote} to place it different.
%
% No page break may occur inside a margin note created with \cs{marginnote}.
%
% \cs{marginnote} is somewhat different from \cs{marginpar} if used immediate
% after \cs{item}. This is not a bug, it's a feature!
%
% With math \cs{marginnote} may work or may not depending on the math
% environment.
%
% If you are using \XeTeX{}, PDF\LaTeX{} since version~1.40 or PDF\LaTeX{}
% before version~1.40 with PDF output and the horizontal position of
% a margin note is wrong, do one more PDF\LaTeX{} run.
%
% Sometimes lines are stretched vertically using \cs{marginnote}, e.g.\ if
% you're using \cs{marginnote} at a list \emph{and} upper case umlauts like
% ``\"U'' or if \verb|\lineskiplimit>0pt|. In this case
% \verb|\lineskiplimit=0pt| or \verb|\lineskiplimit=-\maxdimen|, or one of the
% options may help.
%
% You should not use \cs{marginnote} at the optional argument of \cs{item}.
%
% If \cs{if@twocolumn} is \cs{iftrue}, e.g., because you are using option
% |twocolumn| or command |\twocolumn|, \cs{marginnote} does decide whether the
% note should be placed left of the column or right of the columns simply by
% comaring the current horizontal possition with |\columnwidth+\columnsep|. So
% if the current horizontal possition is somewhere in the left columns, the
% note is placed in the left margin. If the current horizontal possition is
% somewhere right of the left column, the note is placed in the right
% margin. However, support for twocolumn mode is as problematic as support for
% reverse margin notes. I do not like it. Maybe it will be changed in
% future. The current support for twocolumn mode has been implemented only
% because of a feature request by Florent Chervet.
%
% \StopEventually{\PrintIndex\PrintChanges}
%
% \section{Implementation}
%
% \iffalse
%<*package>
% \fi
%
% \changes{v1.3}{2018/04/13}{$\varepsilon$-\TeX{} removed}^^A
%
% \changes{v1.3}{2018/04/13}{early \cs{pdfsavepos}/\cs{savepos} test}^^A
% \begin{macro}{\mn@savepos}
% \changes{v1.3}{2018/04/13}{new internal command}
% \begin{macro}{\mn@lastxpos}
% \changes{v1.3}{2018/04/13}{new internal command}
% Since version 1.3 \texttt{marginnote} does need either \cs{pdfsavepos} and
% \cs{pdflastxpos} or \cs{savepos} and \cs{lastxpos} and does not longer
% support engines without these primitives. All these engines also provide
% $\varepsilon$-\TeX{} extensions. So we do not longer need an explicite
% $\varepsilon$-\TeX{} test.
%    \begin{macrocode}
\begingroup
  \@ifundefined{pdfsavepos}{%
    \@ifundefined{savepos}{%
      \PackageError{marginnote}{%
        neither \string\pdfsavepos\space nor \string\savepos\space
        available
      }{%
        Package `marginnote' depends on extended features of
        PDFLaTeX,\MessageBreak
        LuaLaTeX or XeLaTeX. It does not work without those
        feature.\MessageBreak
        If you'd continue the package will not provide any feature.
      }%
      \aftergroup\endinput
    }{%
      \@ifundefined{lastxpos}{%
        \PackageError{marginnote}{%
          \string\savepos\space but not \string\lastxpos\space
          available
        }{%
          Package `marginnote' depends on extended features of
          PDFLaTeX,\MessageBreak
          LuaLaTeX or XeLaTeX. It does not work without those
          feature.\MessageBreak
          If you'd continue the package will not provide any feature.
        }%
        \aftergroup\endinput
      }{%
        \global\let\mn@savepos\savepos
        \global\let\mn@lastxpos\lastxpos
        \global\let\mn@pagewidth\pagewidth
      }%
    }%
  }{%
    \@ifundefined{pdflastxpos}{%
      \PackageError{marginnote}{%
        \string\pdfsavepos\space but not \string\pdflastxpos\space
        available
      }{%
        Package `marginnote' depends on extended features of
        PDFLaTeX,\MessageBreak
        LuaLaTeX or XeLaTeX. It does not work without those
        feature.\MessageBreak
        If you'd continue the package will not provide any feature.
      }%
      \aftergroup\endinput
    }{%
      \global\let\mn@savepos\pdfsavepos
      \global\let\mn@lastxpos\pdflastxpos
      \global\let\mn@pagewidth\pdfpagewidth
    }%
  }%
\endgroup
%    \end{macrocode}
% \end{macro}
% \end{macro}
% 
% Next declare and process the options.
%
% \begin{macro}{\if@mn@verbose}
% Use verbose output mode by default. But you may change this using option
% \texttt{quiet}.
%    \begin{macrocode}
\newif\if@mn@verbose\@mn@verbosetrue
\DeclareOption{verbose}{\@mn@verbosetrue}
\DeclareOption{quiet}{\@mn@verbosefalse}
%    \end{macrocode}
% \end{macro}
%
% \changes{v1.1e}{2009/06/06}{new options \texttt{fulladjust},
%   \texttt{heightadjust}, \texttt{depthadjust}, and \texttt{noadjust}}
% \begin{macro}{\mn@strut}
% \changes{v1.1e}{2009/06/06}{new (semi internal)}
% The package needs to adjust the margin note at the current line. Sometimes
% this causes extra vertical line spacing. To avoid this you may redefine
% \cs{mn@strut}. The default value is \cs{strut}.
%    \begin{macrocode}
\newcommand*{\mn@strut}{}
%    \end{macrocode}
% \begin{macro}{\mn@zbox}
%   \changes{v1.1b}{2009/02/16}{new (internal)}
% This macro is used to set a horizontal box without height, depth and width.
%    \begin{macrocode}
\newcommand{\mn@zbox}[1]{}
%    \end{macrocode}
% The options do redefine both, \cs{mn@strut} and \cs{mn@zbox}.
%    \begin{macrocode}
\DeclareOption{fulladjust}{%
  \renewcommand*{\mn@strut}{\strut}%
  \renewcommand{\mn@zbox}[1]{%
    \bgroup
      \setbox\@tempboxa\vbox{#1}%
      \ht\@tempboxa\ht\strutbox
      \dp\@tempboxa\dp\strutbox
      \wd\@tempboxa\z@
      \box\@tempboxa
    \egroup
  }%
}
\DeclareOption{heightadjust}{%
  \renewcommand*{\mn@strut}{\begingroup\dp\strutbox\z@\strut\endgroup}%
  \renewcommand{\mn@zbox}[1]{%
    \bgroup
      \setbox\@tempboxa\vbox{#1}%
      \ht\@tempboxa\ht\strutbox
      \dp\@tempboxa\dp\z@
      \wd\@tempboxa\z@
      \box\@tempboxa
    \egroup
  }%
}
\DeclareOption{depthadjust}{%
  \renewcommand*{\mn@strut}{\begingroup\ht\strutbox\z@\strut\endgroup}%
  \renewcommand{\mn@zbox}[1]{%
    \bgroup
      \setbox\@tempboxa\vbox{#1}%
      \ht\@tempboxa\ht\z@
      \dp\@tempboxa\dp\strutbox
      \wd\@tempboxa\z@
      \box\@tempboxa
    \egroup
  }%
}
\DeclareOption{noadjust}{%
  \renewcommand*{\mn@strut}{\relax}%
  \renewcommand{\mn@zbox}[1]{%
    \bgroup
      \setbox\@tempboxa\vbox{\kern-\ht\strutbox #1}%
      \ht\@tempboxa\ht\z@
      \dp\@tempboxa\dp\z@
      \wd\@tempboxa\z@
      \box\@tempboxa
    \egroup
  }%
}
%    \end{macrocode}
% \end{macro}
% \end{macro}
%
% \changes{v1.4}{2018/07/01}{new options \texttt{parboxrestore} (default) and
%   \texttt{noparboxrestore}}^^A
% \begin{macro}{\mn@parboxrestore}
% \changes{v1.4}{2018/07/01}{new internal command}^^A
% We can either use \cs{@parboxrestore} inside the margin notes or dont use
% it. I would recommend to use it, so this will be the new default.
%    \begin{macrocode}
\newcommand*{\mn@parboxrestore}{}
\DeclareOption{parboxrestore}{%
  \renewcommand*{\mn@parboxrestore}{\@parboxrestore}%
}
\DeclareOption{noparboxrestore}{%
  \renewcommand*{\mn@parboxrestore}{}%
}
%    \end{macrocode}
% \end{macro}
%
%    \begin{macrocode}
\ExecuteOptions{verbose,fulladjust,parboxrestore}
\ProcessOptions\relax
%    \end{macrocode}
%
% \begin{macro}{\newmarginnote}
% We need a macro to define a new note at the \texttt{aux} file. This will
% be done using the mechanism of \LaTeX{} that is used for
% \cs{newlabel}. But we use another prefix. This will result in the usual
% ``Labels(s) may have changed. Rerun to get cross-references right.'' if a
% margin note is new or have moved to another page.
%    \begin{macrocode}
\newcommand*{\newmarginnote}{\@newl@bel{mn}}
%    \end{macrocode}
% \end{macro}
%
% \begin{macro}{\if@mn@pdfmode}
% \changes{v1.1}{2006/10/23}{new switch}^^A
% \changes{v1.1a}{2008/11/10}{PDF\TeX\ since 1.40 allows \cs{pdfsavepos} in
%   DVI mode too}^^A
% \changes{v1.1b}{2009/02/16}{if level fixed}^^A
% \changes{v1.1c}{2009/02/26}{\protect\XeTeX has working \cs{pdflastxpos}^^A
%   but no \cs{pdftexversion}}^^A
% \changes{v1.2}{2016/06/02}{addition for lua\TeX{} from 0.85}^^A
% \changes{v1.3}{2018/04/13}{removed}^^A
% \begin{macro}{\@mn@mode@prefix}
% \changes{v1.2}{2016/06/02}{(new (internal)}^^A
% \changes{v1.3}{2018/04/13}{removed}^^A
% \end{macro}
% \end{macro}
%
% \begin{macro}{\marginnotetextwidth}
% \changes{v1.1}{2006/10/23}{new macro}
% Some environments change \cs{textwidth}. But at PDF mode we need to know the
% real text width to find the right margin. So we use our own text width
% macro. Sometimes it may be useful if the user can set it up. Because of
% this it is a user command.
%    \begin{macrocode}
\newcommand*{\marginnotetextwidth}{}
\let\marginnotetextwidth\textwidth
\AtBeginDocument{\edef\marginnotetextwidth{\the\textwidth}}
%    \end{macrocode}
% \end{macro}
%
% \begin{macro}{\@mn@margintest}
% \changes{v1.1}{2006/10/23}{new PDF mode feature}
% \begin{macro}{\@mn@thispage}
% \begin{macro}{\@mn@atthispage}
% \begin{macro}{\@mn@currpage}
% \changes{v1.1}{2006/10/23}{new (internal)}
% \begin{macro}{\@mn@currxpos}
% \changes{v1.1}{2006/10/23}{new (internal)}
% \begin{macro}{\mn@abspage}
% Macro \cs{@mn@margintest} does the complete test, which margin to use. The
% result may be found at \cs{if@tempswa}. To avoid changes on the last page
% if there is a new note on the first page, try to count the notes by
% page. We know that this can not be successful, but never the less it may
% be a good try. \cs{@mn@thispage} saves the page number of the last usage
% of \cs{@mn@margintest}. \cs{@mn@atthispage} saves the number of margin
% note at this page. But we need to know the absolut page number to do
% this. So we increase the absolut page number \texttt{mn@abspage} at every
% \cs{@outputpage}. \cs{@mn@currpage} is the page from the page label if
% found. \cs{@mn@currxpos} is the real $x$ position may be written with the
% page label and used to calculate the correct horizontal offset.
%    \begin{macrocode}
\newcommand*{\@mn@thispage}{}
\newcommand*{\@mn@currpage}{}
\newcommand*{\@mn@currxpos}{}
\newcounter{mn@abspage}
\AtBeginDocument{\setcounter{mn@abspage}{1}%
  \g@addto@macro\@outputpage{%
    \stepcounter{mn@abspage}%
%    \end{macrocode}
% \changes{v1.4a}{2018/07/21}{workaround for screwball page order}^^A
% From version 1.4a there is a workaround for consecutive odd pages or
% consecutive even pages in a twoside document.
%    \begin{macrocode}
    \ifodd\value{mn@abspage}%
      \ifodd\value{page}%
      \else
        \if@twoside
          \begingroup
            \advance\c@page\m@ne
            \PackageWarningNoLine{marginnote}{%
              Consecutive odd pages found.\MessageBreak
              Note, it is not recommended to use consecutive\MessageBreak
              odd pages in a double-ended document.\MessageBreak
              The pages of your document should always\MessageBreak
              be a sequence: odd-even-odd-even-...\MessageBreak
              Maybe you've forgotten a
              \@ifundefined{KOMAClassName}%
                           {\string\cleardoublepage}%
                           {\string\cleardoubleoddpage}
              before\MessageBreak 
              changing the page numbering on page \thepage
            }%
          \endgroup
        \fi
        \PackageInfo{marginnote}{Using workaround for absolute page number}%
        \stepcounter{mn@abspage}%
      \fi
    \else
      \ifodd\value{page}%
        \if@twoside
          \begingroup
            \advance\c@page\m@ne
            \PackageWarningNoLine{marginnote}{%
              Consecutive even pages found.\MessageBreak
              Note, it is not recommended to use consecutive\MessageBreak
              even pages in a double-ended document.\MessageBreak
              The pages of your document should always\MessageBreak
              be a sequence: odd-even-odd-even-...\MessageBreak
              Maybe you've forgotten a
              \@ifundefined{KOMAClassName}%
                           {\string\cleardoublepage}%
                           {\string\cleardoubleevenpage}
              before\MessageBreak 
              changing the page numbering on page \thepage
            }%
          \endgroup
        \fi
        \PackageInfo{marginnote}{Using workaround for absolute page number}%
        \stepcounter{mn@abspage}%
      \fi
    \fi   
  }%
}
\newcommand*{\@mn@margintest}{%
%    \end{macrocode}
% \changes{v1.2}{2016/06/02}{addition for lua\TeX{} from 0.85}^^A
%   Number of the next margin note at this page.
%    \begin{macrocode}
  \expandafter\ifx\csname @mn@thispage\endcsname\@empty
    \gdef\@mn@atthispage{1}%
  \else\expandafter\ifnum \@mn@thispage=\value{mn@abspage}%
      \begingroup
        \@tempcnta\@mn@atthispage\advance\@tempcnta by \@ne
        \xdef\@mn@atthispage{\the\@tempcnta}%
      \endgroup
    \else
      \gdef\@mn@atthispage{1}%
    \fi
  \fi
  \xdef\@mn@thispage{\themn@abspage}%
%    \end{macrocode}
% Use the number of the page and the number of the margin note at this page
% to save the real number of this page at the \texttt{aux} file. At PDF mode
% save the current $x$ position too.
% \changes{v1.3}{2018/04/13}{use new internals \cs{mn@savepos} and
%   \cs{mn@lastxpos}}^^A
% \changes{v1.3}{2018/04/13}{non PDF mode removed}^^A
%    \begin{macrocode}
  \let\@mn@currpage\relax
  \let\@mn@currxpos\relax
  \mn@savepos
  \protected@write\@auxout{\let\themn@abspage\relax}{%
    \string\newmarginnote{note.\@mn@thispage.\@mn@atthispage}{%
      {\themn@abspage}{\noexpand\number\mn@lastxpos sp}}%
  }%
%    \end{macrocode}
% If the margin note label was not defined, it seems to be new. In this case
% the absolut page number will be used for the test instead of the saved
% real page number.
%    \begin{macrocode}
  \expandafter\ifx\csname mn@note.\@mn@thispage.\@mn@atthispage\endcsname\relax
%    \end{macrocode}
% If we are not in two side mode, we are on a odd page.
%    \begin{macrocode}
    \if@twoside
      \if@mn@verbose
        \PackageInfo{marginnote}{Suggest that margin
          note \@mn@thispage.\@mn@atthispage\space will be on\MessageBreak
          absolute page \themn@abspage.\MessageBreak
          This may be wrong}%
      \fi
      \ifodd\value{mn@abspage}\@tempswatrue\else\@tempswafalse\fi
    \else
      \if@mn@verbose
        \PackageInfo{marginnote}{right page because not two side mode}%
      \fi
      \@tempswatrue
    \fi
  \else
    \edef\@mn@currpage{\csname
      mn@note.\@mn@thispage.\@mn@atthispage\endcsname}%
    \edef\@mn@currxpos{\expandafter\@secondoftwo\@mn@currpage}%
%    \end{macrocode}
% \changes{v1.1d}{2009/05/06}{take care of \cs{hoffset}}^^A
% Ulrike Fischer suggested a simple change to take care of \cs{hoffset},
% e.g., using package \textsf{crop}.
% \changes{v1.1d}{2009/05/06}{take care of \cs{pdfhorigin}}^^A
% We use this occasion to take care of \cs{pdfhorigin}, too.
% \changes{v1.2a}{2016/10/21}{redefine \cs{@mn@currxpos} only if not empty}^^A
% If \cs{@mn@currxpos} is not empty here, it should be corrected by
% \cs{hoffset} and maybe by \cs{pdfhorigin}.
%    \begin{macrocode}
    \ifx\@mn@currxpos\@empty\else
      \edef\@mn@currxpos{\the\dimexpr \@mn@currxpos -\hoffset\relax}%
      \begingroup\expandafter\expandafter\expandafter\endgroup
      \expandafter\ifx\csname pdfhorigin\endcsname\relax\else
        \begingroup\expandafter\expandafter\expandafter\endgroup
        \expandafter\ifx\csname pdfoutput\endcsname\relax
          \begingroup\expandafter\expandafter\expandafter\endgroup
          \expandafter\ifx\csname outputmode\endcsname\relax\else
            \ifnum \outputmode=1 %
              \edef\@mn@currxpos{\the\dimexpr \@mn@currxpos -\pdfhorigin
                +1in\relax}%
            \fi
          \fi
        \else
          \ifnum \pdfoutput=1 %
            \edef\@mn@currxpos{\the\dimexpr \@mn@currxpos -\pdfhorigin 
              +1in\relax}%
          \fi
        \fi
      \fi
%    \end{macrocode}
% \changes{v1.2b}{2017/04/22}{\textsf{bidi} code added}^^A
% \changes{v1.3}{2018/04/13}{\texttt{twocolumn} test added}^^A
% If you are using package \textsf{bidi} and RTL mode is active, the position
% is from right instead of left. So we have to substract \cs{@mn@currxpos}
% from \cs{pdfpagewidth} (or \cs{pagewidth} using Lua\TeX, but this cannot be,
% because \textsf{bidi} is not Lua\TeX-compatible).
%    \begin{macrocode}
      \ifdefined\mn@pagewidth
        \@mn@if@RTL{%
          \PackageInfo{marginnote}{Margin note
            \@mn@thispage.\@mn@atthispage\space in RTL mode}%
          \edef\@mn@currxpos{%
            \the\dimexpr\mn@pagewidth-\@mn@currxpos\relax
          }%
        }{}%
      \fi
    \fi
    \edef\@mn@currpage{\expandafter\@firstoftwo\@mn@currpage}%
    \if@mn@verbose
      \PackageInfo{marginnote}{Margin note \@mn@thispage.\@mn@atthispage\space
        is on absolute page \@mn@currpage}%
    \fi
    \if@twoside
      \ifodd\@mn@currpage\relax
        \@tempswatrue
        \if@twocolumn
          \ifdim \@mn@currxpos
                 < \dimexpr\oddsidemargin+\columnwidth+\columnsep\relax
            \@tempswafalse
          \fi
        \fi
      \else
        \@tempswafalse
        \if@twocolumn
          \ifdim\@mn@currxpos>\dimexpr\evensidemargin+\columnwidth\relax
            \@tempswatrue
          \fi
        \fi  
      \fi
    \else
      \if@mn@verbose
        \PackageInfo{marginnote}{right page because not two side mode}%
      \fi
      \@tempswatrue
      \if@twocolumn
        \ifdim \@mn@currxpos
               < \dimexpr\oddsidemargin+\columnwidth+\columnsep\relax
          \@tempswafalse
        \fi
      \fi
    \fi  
  \fi
}
%    \end{macrocode}
% \begin{macro}{@mn@ifRTL}
% \changes{v1.2b}{2017/04/22}{new internal}
% Test, whether or not \cs{if@RTL} exists and is true or false.
%    \begin{macrocode}
\newcommand*{\@mn@if@RTL}{%
  \begingroup\expandafter\expandafter\expandafter\endgroup
  \expandafter\ifx\csname if@RTL\endcsname\iftrue
    \expandafter\@firstoftwo
  \else
    \expandafter\@secondoftwo
  \fi
}
%    \end{macrocode}
% \end{macro}
% \end{macro}
% \end{macro}
% \end{macro}
% \end{macro}
% \end{macro}
% \end{macro}
%
% \begin{macro}{\marginnote}
% \begin{macro}{\@mn@marginnote}
% \begin{macro}{\@mn@@marginnote}
% \changes{v1.1g}{2011/04/11}{missing \cs{long} added}
% \begin{macro}{\@mn@@@marginnote}
% \changes{v1.1}{2006/10/23}{new PDF mode feature}
% \changes{v1.1g}{2011/04/11}{missing \cs{long} added}
% Command \cs{marginnote} is the main macro of the package. The others are
% helpers to manage the optional arguments.
%    \begin{macrocode}
\newcommand*{\marginnote}{%
  \@dblarg\@mn@marginnote
}
\newcommand{\@mn@marginnote}[2][]{%
  \ifhmode
    \@bsphack
    \begingroup
    \ifdim\@savsk>\z@\else
      \def\:{\@xifnch}\expandafter\def\: { \futurelet\@let@token\@ifnch}%
    \fi
  \else
    \begingroup
  \fi
  \@ifnextchar [{\@mn@@marginnote[{#1}]{#2}}{\@mn@@marginnote[{#1}]{#2}[\z@]}%
}
\newcommand{\@mn@@marginnote}{}
\long\def\@mn@@marginnote[#1]#2[#3]{%
  \endgroup
%    \end{macrocode}
% In horizontal mode the space hack of the \LaTeX{} kernel will be used. In
% vertical mode this should not be used.
%    \begin{macrocode}
  \ifhmode
    \@mn@@@marginnote[{#1}]{#2}[{#3}]%
    \@esphack
  \else
    \@mn@@@marginnote[{#1}]{#2}[{#3}]%
  \fi
}
\newcommand{\@mn@@@marginnote}{}
\long\def\@mn@@@marginnote[#1]#2[#3]{%
%    \end{macrocode}
% \changes{v1.1b}{2009/02/16}{use \cs{mn@vadjust} instead of \cs{vadjust}}%
% \changes{v1.1e}{2009/06/06}{use \cs{mn@strut} instead of \cs{strut}}%
% All changes (but change of counters that are global because of using the
% \LaTeX{} commands to change them an \cs{gdef} and \cs{xdef}) should be
% local. In h-mode a \cs{strut} will be used to fix base line. The margin
% note will be put to vertical list using \cs{vadjust}. This also means that
% wie are one line to deep. This will be corrected later using negative kern.
% In v-mode wie use a special kind of vbox to simply set everything. Math
% mode should behave like v-mode. And if we are just after an item we have
% to leave v-mode first.
%    \begin{macrocode}
  \begingroup
    \ifmmode\mn@strut\let\@tempa\mn@vadjust\else
      \if@inlabel\leavevmode\fi
      \ifhmode\mn@strut\let\@tempa\mn@vadjust\else\let\@tempa\mn@vlap\fi
    \fi
    \@tempa{%
%    \end{macrocode}
% Everything will be put upwards using a \cs{vbox} with zero height and depth
% and \cs{vss}. At this box the margin test will be done. If
% \cs{reversemarginpar} was used, the logic reverses. Then the note will be
% places to the margin.
%    \begin{macrocode}
      \vbox to\z@{%
        \vss
        \@mn@margintest
        \if@reversemargin\if@tempswa
            \@tempswafalse
          \else
            \@tempswatrue
        \fi\fi
        \if@tempswa
          \rlap{%
%    \end{macrocode}
% If \cs{@mn@currxpos} is neither \cs{relax} nor empty it is the real 
% current $x$ position of the last PDF\LaTeX{} run and may be used to
% calculate the real horizontal offset.
%    \begin{macrocode}
            \if@mn@verbose
              \PackageInfo{marginnote}{xpos seems to be \@mn@currxpos}%
            \fi
            \begingroup
              \ifx\@mn@currxpos\relax\else\ifx\@mn@currxpos\@empty\else
                  \kern-\dimexpr\@mn@currxpos\relax
              \fi\fi
              \ifx\@mn@currpage\relax
                \let\@mn@currpage\@ne
              \fi
              \if@twoside\ifodd\@mn@currpage\relax
                  \kern\oddsidemargin
                \else
                  \kern\evensidemargin
                \fi
              \else
                \kern\oddsidemargin
              \fi
              \kern 1in
            \endgroup
            \kern\marginnotetextwidth\kern\marginparsep
            \vbox to\z@{\kern\marginnotevadjust\kern #3
              \vbox to\z@{%
                \hsize\marginparwidth
%    \end{macrocode}
% \changes{v1.1g}{2011/04/11}{set \cs{linewidth}}
%    \begin{macrocode}
                \linewidth\hsize
%    \end{macrocode}
% Here's the correction of the vertical position. The remain is simple.
% \changes{v1.1i}{2012/03/29}{\cs{strut} moved to fix hyphenation (thanks to
%   Ulrike Fischer)}
% \changes{v1.1i}{2012/03/29}{\cs{ignorespaces} added}^^A
% \changes{v1.4}{2018/07/01}{\cs{mn@parboxrestore} added}^^A
%    \begin{macrocode}
                \kern-\parskip
                \mn@parboxrestore
                \marginfont\raggedrightmarginnote\strut\hspace{\z@}%
                \ignorespaces#2\endgraf
                \vss}%
              \vss}%
          }%
        \else
%    \end{macrocode}
% Using the left margin.
% \changes{v1.1f}{2010/01/05}{missing usage of \cs{marginnotevadjust} on
%   left margin fixed}
%    \begin{macrocode}
          \llap{%
            \vbox to\z@{\kern\marginnotevadjust\kern #3
              \vbox to\z@{%
                \hsize\marginparwidth
%    \end{macrocode}
% \changes{v1.1g}{2011/04/11}{set \cs{linewidth}}
%    \begin{macrocode}
                \linewidth\hsize
%    \end{macrocode}
% Same like above for left margins.
%    \begin{macrocode}
                \kern-\parskip
                \mn@parboxrestore
                \marginfont\raggedleftmarginnote\strut\hspace{\z@}%
                \ignorespaces#1\endgraf
                \vss
              }%
              \vss
            }%
            \if@mn@verbose
              \PackageInfo{marginnote}{xpos seems to be \@mn@currxpos}%
            \fi
            \begingroup
              \ifx\@mn@currxpos\relax\else\ifx\@mn@currpos\@empty\else
                  \kern\@mn@currxpos
              \fi\fi
              \ifx\@mn@currpage\relax
                \let\@mn@currpage\@ne
              \fi
              \if@twoside\ifodd\@mn@currpage\relax
                  \kern-\oddsidemargin
                \else
                  \kern-\evensidemargin
                \fi
              \else
                \kern-\oddsidemargin
              \fi
              \kern-1in
            \endgroup
            \kern\marginparsep
          }%
        \fi
      }%
    }%
  \endgroup
}
%    \end{macrocode}
% \end{macro}
% \end{macro}
% \end{macro}
% \end{macro}
%
% \begin{macro}{\marginnoterightadjust}
% \changes{v1.3}{2018/04/13}{removed}^^A
% \begin{macro}{\marginnoteleftadjust}
% \changes{v1.3}{2018/04/13}{removed}^^A
% \end{macro}
% \end{macro}
%
% \begin{macro}{\marginnotevadjust}
% This may be used to define an automatic vertical adjust. The defaul tis
% zero. Values greater than zero will move the margin note down, values less
% than zero will move the margin note up.
%    \begin{macrocode}
\newcommand*{\marginnotevadjust}{}
\let\marginnotevadjust\z@
%    \end{macrocode}
% \end{macro}
%
% \begin{macro}{\mn@vlap}
% This macro is used to set a vertical box without size at vertical mode.
%    \begin{macrocode}
\newcommand{\mn@vlap}[1]{%
  \setbox\@tempboxa\vbox to \ht\strutbox{#1\vss}%
  \box\@tempboxa\vskip-\baselineskip
}
%    \end{macrocode}
% \end{macro}
%
% \begin{macro}{\mn@vadjust}
% \changes{v1.1b}{2009/02/16}{new (internal)}
% This macro is used to set a vertical box at horizontal mode.
%    \begin{macrocode}
\newcommand{\mn@vadjust}[1]{%
  \mn@zbox{\kern-\parskip
    \leavevmode\vadjust{#1}%
    \kern\parskip
  }%
}
%    \end{macrocode}
% \end{macro}
%
% \begin{macro}{\marginfont}
% \changes{v1.0a}{2006/02/06}{Use \cs{providecommand} to define it.}
% \begin{macro}{\raggedleftmarginnote}
% \begin{macro}{\raggedrightmarginnote}
%   These are very simple. A class may also define \cs{marginfont}. Use this
%   if available. I don't use \cs{let} for the definitions of the ragged
%   macros, so the meaning may change loading e.g. package \textsf{ragged2e}.
%    \begin{macrocode}
\providecommand*{\marginfont}{}
\newcommand*{\raggedleftmarginnote}{\raggedleft}
\newcommand*{\raggedrightmarginnote}{\raggedright}
%    \end{macrocode}
% \end{macro}
% \end{macro}
% \end{macro}
%
% \Finale
%
\endinput
%
% end of `marginnote.dtx'
%
% \iffalse
%%% Local Variables:
%%% mode: doc-tex
%%% TeX-master: t
%%% End:
% \fi
